\documentclass[12pt, a4paper]{article}

\usepackage[utf8]{inputenc}
\usepackage[T1]{fontenc}
\usepackage[brazil]{babel}
\usepackage{geometry}
\usepackage{array}
\usepackage{longtable}

% Configuração das margens
\geometry{
    a4paper,
    left=1.5cm,
    right=1.5cm,
    top=2cm,
    bottom=2cm
}

\begin{document}

\begin{longtable}{|p{0.25\linewidth}|p{0.7\linewidth}|}
\hline
\multicolumn{2}{|c|}{\textbf{Identificação do Problema}} \\
\hline
\textbf{Tema} & Arquitetura de Sistema Multiagentes com Blockchain para Negociação de Contratos de Serviços em Nuvem Distribuída \\
\hline
\textbf{Problema} & A predominância de provedores centralizados de serviços em nuvem limita a flexibilidade, aumenta os custos e reduz a transparência na alocação de recursos computacionais. Além disso, a ausência de mecanismos automatizados e confiáveis para negociação entre múltiplas partes impede o desenvolvimento de ecossistemas descentralizados e colaborativos. Essa centralização cria dependência crítica e compromete a soberania dos usuários e pequenos provedores sobre sua infraestrutura. \\
\hline
\textbf{Pergunta de Pesquisa} & Como um sistema multiagentes, utilizando blockchain, pode simular a negociação de contratos de serviços de infraestrutura da nuvem distribuída? \\
\hline
\textbf{Hipótese} & É possível projetar e simular um sistema multiagentes que negocie contratos de nuvem em blockchain, otimizando a alocação e o gerenciamento de recursos e aumentando a transparência e acessibilidade. \\
\hline
\textbf{Justificativa da Hipótese} & A dependência de infraestruturas centralizadas como AWS, Meta e Baidu traz riscos de centralização, falhas únicas, privacidade, segurança, altos custos e falta de controle para usuários, criando um oligopólio econômico e social vulnerável a abusos, o que torna essencial explorar alternativas baseadas em blockchain para descentralizar e aumentar a transparência, segurança e resiliência. \\
\hline
\textbf{Objetivo} & Definir arquitetura do sistema multiagentes e o modelo para contrato de modo a precificar os recursos computacionais de maneira sensata. \\
\hline
\textbf{Objetivo Específico} & Projetar e implementar simulações de provas de conceito de sistemas multiagentes para negociar contratos de compra e venda de serviços de infraestrutura de nuvem distribuída usando blockchain, manter a contabilidade de recursos disponíveis e futuros, além de pesquisar protocolos de consenso para acelerar acordos registrados via smart contracts. \\
\hline
\textbf{Método} & Pesquisa Bibliográfica e Revisão Sistemática da Literatura, Modelagem e Desenvolvimento (Definição dos Agentes, Desenho da Arquitetura do Sistema, Modelagem do Contrato Inteligente, Desenvolvimento do Modelo de Precificação), Simulação Computacional (Escolha da Plataforma, Implementação, Execução, Coleta e Análise de Dados), Avaliação e Validação. \\
\hline
\textbf{Classificação da pesquisa} & Natureza: Aplicada; Objetivos: Exploratória; Abordagem: Quantitativa; Técnica: Simulação. \\
\hline
\caption{Identificação do Problema de Pesquisa.}
\label{tab:pesquisa}
\end{longtable}

\end{document}