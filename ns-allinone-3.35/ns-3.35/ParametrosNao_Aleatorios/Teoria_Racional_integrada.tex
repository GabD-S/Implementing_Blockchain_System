\documentclass[12pt, a4paper]{article}

% --- PREÂMBULO ---
\usepackage[utf8]{inputenc}
\usepackage[T1]{fontenc}
\usepackage[portuguese]{babel}
\usepackage{geometry}
\usepackage{amsmath}
\usepackage{amssymb}
\usepackage{xcolor}
\usepackage{fancyhdr}
\usepackage{titlesec}
\usepackage{setspace}
\usepackage{tocloft}

% Configuração de margens
\geometry{
 a4paper,
 total={170mm,257mm},
 left=25mm,
 right=25mm,
 top=25mm,
 bottom=25mm,
}

% Espaçamento entre linhas
\onehalfspacing

% Configuração de cores
\definecolor{darkblue}{RGB}{0, 51, 102}
\definecolor{lightblue}{RGB}{230, 242, 255}
\definecolor{accentcolor}{RGB}{0, 102, 204}

% Formatação de títulos
\titleformat{\section}
{\normalfont\Large\bfseries\color{darkblue}}
{\thesection}{1em}{}[\titlerule]

\titleformat{\subsection}
{\normalfont\large\bfseries\color{accentcolor}}
{\thesubsection}{1em}{}

\titleformat{\subsubsection}
{\normalfont\normalsize\bfseries\color{accentcolor}}
{\thesubsubsection}{1em}{}

% Cabeçalho e rodapé
\pagestyle{fancy}
\fancyhf{}
\fancyhead[L]{\small\textcolor{darkblue}{\textbf{Simulação de Mercado}}}
\fancyhead[R]{\small\thepage}
\renewcommand{\headrulewidth}{0.5pt}
\renewcommand{\headrule}{\color{accentcolor}\hrule width\headwidth height\headrulewidth \vskip-\headrulewidth}

% Permite parágrafos com indentação
\setlength{\parindent}{1.25cm}
\setlength{\parskip}{0.3\baselineskip}

% --- INÍCIO DO DOCUMENTO ---
\begin{document}

% ====== CAPA 1 ======
\thispagestyle{empty}
\begin{center}
\vspace*{3cm}

{\Huge\bfseries\color{darkblue} Simulação de Mercado}

\vspace{0.5cm}

{\Large\color{accentcolor} Duas Abordagens Teóricas}

\vspace{3cm}

{\Large\bfseries Abordagem 1}

{\huge\color{accentcolor} Teoria Racional Integrada}

\vspace{2cm}

{\large\color{darkblue} Fundamentos Teóricos e Modelagem de Agentes Econômicos}

\vspace{4cm}

\end{center}

\newpage

% ====== RESUMO 1 ======
\section*{Resumo da Abordagem 1}
\addcontentsline{toc}{section}{Resumo da Abordagem 1}

\noindent\fcolorbox{accentcolor}{lightblue}{%
\parbox{\textwidth}{
\textbf{Objetivo:} Esta primeira abordagem estabelece os fundamentos teóricos para modelar o comportamento de agentes em simulações de mercado, integrando a Teoria da Utilidade Esperada, Teoria dos Jogos e conceitos comportamentais clássicos.

\vspace{0.3cm}

\textbf{Foco:} Desenvolvimento de uma base sólida para entender como agentes econômicos tomam decisões racionais e como essas decisões podem ser modeladas em ambientes simulados.

\vspace{0.3cm}

\textbf{Aplicação:} Ideal para cenários onde a compreensão das mecânicas econômicas fundamentais é necessária antes de introduzir complexidades comportamentais.
}}

\vspace{1.5cm}

\section{Como Aplicar Estas Teorias na Simulação}

Para modelar o comportamento dos agentes e as falhas realistas do mercado (como ``preço alto'' ou ``reputação baixa''), a simulação deve integrar as teorias econômicas de forma estruturada.

\subsection{Fundamentação Teórica}

A modelagem de agentes em simulações de mercado repousa em três pilares teóricos principais:

\begin{enumerate}
    \item \textbf{Teoria da Utilidade Esperada (EUT)} --- Fornece a base para entender como agentes racionais avaliam alternativas e tomam decisões sob incerteza. Embora limitada em sua capacidade de descrever comportamentos reais, serve como ponto de referência para comparações com modelos comportamentais.

    \item \textbf{Teoria dos Jogos} --- Permite modelar interações estratégicas entre agentes, particularmente entre provedores em ambientes competitivos. O conceito de Equilíbrio de Nash é fundamental para prever comportamentos de preço e estratégias de mercado.

    \item \textbf{Economia Comportamental} --- Incorpora a compreensão de que agentes reais desviam da racionalidade perfeita, introduzindo elementos de psicologia econômica e tomada de decisão limitada.
\end{enumerate}

\subsection{Implementação para Compradores}

\noindent\textbf{Use a Teoria da Utilidade Esperada (EUT)} para seus agentes \texttt{Compradores} como baseline racional. Humanos reais não se comportam assim, como demonstrado por múltiplos paradoxos e experimentos. Esta abordagem oferece uma referência para comparação futura com modelos mais sofisticados.

A utilidade esperada de um comprador ao aceitar um preço $P$ é expressa por:

\begin{equation}
U = \sum_{i} p_i \cdot u(x_i)
\end{equation}

\noindent onde $p_i$ é a probabilidade de cada resultado $x_i$ e $u(x_i)$ é a utilidade associada.

\subsection{Implementação para Provedores}

\noindent\textbf{Use a Teoria dos Jogos (Equilíbrio de Nash)} para modelar os \texttt{Provedores}. Eles estão em um jogo competitivo onde cada provedor define seu preço ($P_{\text{provedor}}$) com base no preço esperado dos concorrentes ($P_{\text{concorrentes}}$) para maximizar lucro:

\begin{equation}
\text{Lucro} = (P_{\text{provedor}} - \text{Custo}) \cdot \pi(P_{\text{provedor}} \le P_{\text{comprador}})
\end{equation}

\noindent onde $\pi$ representa a probabilidade de aceitação do comprador.

Em equilíbrio de Nash, nenhum provedor pode melhorar sua posição alterando unilateralmente seu preço, dada a estratégia dos concorrentes.

\subsection{Identificação de Falhas de Mercado}

As negociações falham no mundo real quando há desalinhamento entre as expectativas dos agentes e a realidade do mercado. Esta abordagem racional identifica falhas através de:

\begin{itemize}
    \item Assimetria de informação entre compradores e provedores
    \item Divergência entre o preço de equilíbrio teórico e ofertas concretas
    \item Ineficiências causadas pela racionalidade limitada dos agentes
\end{itemize}

\newpage

% ====== CAPA 2 ======
\thispagestyle{empty}
\begin{center}
\vspace*{3cm}

{\Huge\bfseries\color{darkblue} Simulação de Mercado}

\vspace{0.5cm}

{\Large\color{accentcolor} Duas Abordagens Teóricas}

\vspace{3cm}

{\Large\bfseries Abordagem 2}

{\huge\color{accentcolor} Teoria Comportamental Avançada}

\vspace{2cm}

{\large\color{darkblue} Integração de Vieses Psicológicos e Dinâmicas Realistas}

\vspace{4cm}

\end{center}

\newpage

% ====== RESUMO 2 ======
\section*{Resumo da Abordagem 2}
\addcontentsline{toc}{section}{Resumo da Abordagem 2}

\noindent\fcolorbox{accentcolor}{lightblue}{%
\parbox{\textwidth}{
\textbf{Objetivo:} Esta segunda abordagem incorpora elementos avançados de economia comportamental, especialmente a Teoria da Perspectiva, para capturar vieses psicológicos reais que afetam as decisões de mercado.

\vspace{0.3cm}

\textbf{Foco:} Modelagem realista de compradores e provedores considerando aversão à perda, pontos de referência individualizados e comportamentos que violam a racionalidade econômica clássica.

\vspace{0.3cm}

\textbf{Aplicação:} Apropriada para simulações que buscam reproduzir falhas realistas de mercado, rejeições de ofertas economicamente justificadas e fenômenos de contágio comportamental.
}}

\vspace{1.5cm}

\section{Aplicação na Simulação}

\subsection{Princípios Fundamentais da Abordagem Comportamental}

A abordagem comportamental avançada reconhece que os agentes reais não são maximizadores racionais perfeitos. Em vez disso, eles operam dentro de limites cognitivos, utilizam heurísticas e são influenciados por vieses psicológicos profundos.

\subsubsection{Implementação para Compradores}

\noindent\textbf{Use a Teoria da Perspectiva (PT)} para modelar seus \texttt{Compradores}. A ``capacidade financeira'' (orçamento) deles deve ser o \textbf{ponto de referência} $v(0)$. A decisão de aceitar um preço do \texttt{Provedor} será baseada em:

\begin{equation}
v(\text{PreçoOfertado} - \text{PontoDeReferência})
\end{equation}

A função de valor na Teoria da Perspectiva é assimétrica:

\begin{equation}
v(x) = \begin{cases}
x^{\alpha} & \text{se } x \geq 0 \text{ (ganhos)} \\
-\lambda(-x)^{\beta} & \text{se } x < 0 \text{ (perdas)}
\end{cases}
\end{equation}

\noindent onde $\lambda > 1$ (tipicamente $\lambda \approx 2.25$) representa a aversão à perda.

\vspace{0.5cm}

\noindent\textbf{Exemplo Prático:} Um comprador com R\$ 500 (capacidade) rejeitará um preço que um comprador com R\$ 5000 (capacidade) aceitaria, mesmo que ambos busquem o mesmo serviço. O primeiro agente enquadra o preço como uma ``perda'' significativa em relação ao seu ponto de referência, enquanto o segundo o vê como trivial. Isso decorre da aversão à perda, não de uma diferença de racionalidade.

\subsubsection{Implementação para Provedores}

\noindent\textbf{Use a Teoria dos Jogos Comportamental} para modelar \texttt{Provedores} que não apenas maximizam lucro, mas também considerem reputação e aceitação social. Um provedor que cobra preços ``injustos'' sofre perda de reputação mesmo que economicamente racional.

A estratégia de preço deve considerar:

\begin{equation}
\text{Payoff}_{\text{provedor}} = \text{Lucro} - \text{Penalidade}_{\text{reputação}}(\text{preço percebido como injusto})
\end{equation}

\subsubsection{Modelagem de Falhas Realistas}

O motivo pelo qual as negociações falham no mundo real é a \textbf{Teoria dos Jogos Comportamental} --- a mistura de Teoria da Perspectiva e Teoria dos Jogos. A falha ocorre quando:

\begin{itemize}
    \item O \texttt{Comprador} (via Teoria da Perspectiva) sente que o preço do \texttt{Provedor} é uma ``perda'' injusta e o \textit{pune} (recusa), mesmo que o preço esteja objetivamente abaixo do seu orçamento, violando o Jogo do Ultimato.
    
    \item O \texttt{Provedor} (via Teoria dos Jogos) define um preço que \textit{seria} racional se o comprador fosse 100\% racional, mas falha ao não considerar a psicologia de ``aversão à perda'' do comprador.
    
    \item A \textbf{punição altruísta} ocorre: compradores rejeitam ofertas injustas para punir provedores gananciosos, mesmo que isso lhes custe pessoalmente.
\end{itemize}

\subsection{Recomendações de Implementação}

\subsubsection{Inicialização e Dados}

Utilize um gerador de população sintética (Monte Carlo ou Latin Hypercube Sampling) para inicializar os agentes compradores e provedores, garantindo reprodutibilidade através de seeds fixas.

\subsubsection{Modelagem de Decisões}

Modele a decisão de escolha dos compradores via Modelo de Utilidade Aleatória (RUM) ajustado com a Teoria da Perspectiva para capturar tanto racionalidade econômica quanto vieses comportamentais.

\subsubsection{Dinâmica Temporal}

Simule a chegada temporal de demandas através de Processos de Hawkes, representando picos de atividade e efeitos de contágio no mercado.

\subsubsection{Gestão de Reputação}

Atualize a reputação dos provedores de forma incremental com base em um modelo Beta-Bayesiano, ajustando a cada transação bem-sucedida ou falha. Para ambientes distribuídos, implemente o cálculo periódico de reputação global via algoritmo EigenTrust, reforçando a confiança sistêmica.

\subsubsection{Calibração e Validação}

Calibre os coeficientes de sensibilidade ao preço, reputação e latência a partir de dados empíricos ou rodadas de simulação inicial para ajustar o comportamento dos agentes.

Utilize funções de utilidade normalizadas e thresholds de reputação individualizados para cada comprador, promovendo diversidade comportamental no sistema.

\subsubsection{Reprodutibilidade}

Salve as distribuições amostradas (como budgets, capacidades e preços) para replicar e comparar cenários futuros de simulação.

\end{document}
