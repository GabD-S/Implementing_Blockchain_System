\documentclass[12pt, a4paper]{article}

% --- PREÂMBULO ---
\usepackage[utf8]{inputenc}
\usepackage[T1]{fontenc}
\usepackage[portuguese]{babel}
\usepackage{geometry}
\usepackage{graphicx} % Para incluir imagens
\usepackage{amsmath}  % Para fórmulas matemáticas
\usepackage{float}    % Para melhor controle de flutuação [H]

% Configuração de margens
\geometry{
 a4paper,
 total={170mm,257mm},
 left=20mm,
 top=20mm,
}

% Permite parágrafos com indentação e espaçamento
\setlength{\parindent}{1.25cm}
\setlength{\parskip}{0.2\baselineskip}

% Título
\title{Análise Comportamental da Simulação de Mercado Descentralizado}
\author{Blockchain-Pesquisa (Análise de Simulação Rust)}
\date{\today}


% --- INÍCIO DO DOCUMENTO ---
\begin{document}

\maketitle

\section{Análise dos Resultados da Simulação}

A análise dos resultados da simulação (arquivo \texttt{comportamental\_resultados.json}) revela uma conclusão clara e fundamental:

\textbf{A simulação, no estado atual, não está modelando um mercado \textit{comportamental}; ela está modelando um mercado perfeitamente \textit{racional} onde um provedor é tão superior que todos os compradores, agindo logicamente, o escolhem.}

Isto é revelado por duas descobertas principais nos dados:
\begin{enumerate}
    \item \textbf{Taxa de Aceitação de 100\%:} Todos os 25 compradores encontraram e aceitaram uma oferta.
    \item \textbf{Monopólio de Mercado:} Todos os 25 compradores escolheram o \texttt{provider\_id: 1}.
\end{enumerate}

\section{Expectativa vs. Realidade}

\begin{itemize}
    \item \textbf{A Expectativa (Teoria Comportamental):} Esperávamos ver uma distribuição de escolhas. Alguns compradores com orçamentos apertados (\texttt{ref\_point} baixo) poderiam \textit{recusar} todas as ofertas, sentindo-as como uma "perda" (resultando em \texttt{accepted: false}). Compradores diferentes poderiam escolher provedores diferentes, talvez alguns preferindo uma \texttt{reputation} (reputação) um pouco maior por um \texttt{price} (preço) um pouco maior.

    \item \textbf{A Realidade (Modelo Racional):} Como o \texttt{provider\_id: 1} tem uma combinação de preço muito baixo (R\$ 0.093/GB) e reputação muito alta (0.867), ele é o "vencedor" objetivo. Para \textit{todos} os compradores, o custo total foi tão baixo em relação ao seu \texttt{ref\_point} (orçamento) que a transação foi sempre percebida como um "ganho" (\texttt{value\_fn} positivo). A aversão à perda, o cerne da Teoria da Perspectiva, nunca foi ativada.
\end{itemize}

\section{Análise Gráfica dos Resultados}

\subsection{Gráfico 1: O Cenário dos Provedores (Por que o Provedor 1 Venceu?)}

O primeiro gráfico plota todos os provedores disponíveis em um gráfico de Preço (Price) vs. Reputação (Reputation). O "provedor ideal" estaria no canto inferior direito (baixo preço, alta reputação).

\begin{figure}[H]
    \centering
    \includegraphics[width=0.9\textwidth]{Teoria_deModulacao/Teoria_comportamental/provedor1_wins.png}
    \caption{Cenário de Mercado: Preço vs. Reputação dos Provedores.}
    \label{fig:landscape}
\end{figure}

\textbf{Conclusão do Gráfico 1:} Como suspeitado, o \textbf{Provedor 1} é um ponto fora da curva. Ele está localizado no "quadrante dourado" (canto inferior direito), possuindo simultaneamente uma das reputações mais altas e um dos preços mais baixos. Dado este cenário, qualquer comprador (racional \textit{ou} comportamental) que valorize tanto o preço quanto a reputação escolherá o Provedor 1, pois ele domina objetivamente as outras opções.

\subsection{Gráfico 2: A Percepção do Comprador}

Estes gráficos mostram a "função de valor" (\texttt{value\_fn} - o resultado da fórmula da Teoria da Perspectiva) em relação ao orçamento e ao armazenamento solicitado.

\begin{figure}[H]
    \centering
    \includegraphics[width=\textwidth]{Teoria_deModulacao/Teoria_comportamental/buyerperception.png}
    \caption{Análise da Percepção de Valor do Comprador.}
    \label{fig:perception}
\end{figure}

\textbf{Conclusão do Gráfico 2:} Estes gráficos confirmam que a fórmula da Teoria da Perspectiva está funcionando \textit{conforme programada}:
\begin{enumerate}
    \item \textbf{Valor vs. Orçamento (Gráfico Esquerdo):} Há uma forte correlação positiva. Quanto maior o orçamento (\texttt{ref\_point}) do comprador, maior o "ganho" percebido (\texttt{value\_fn}).
    \item \textbf{Valor vs. Armazenamento (Gráfico Direito):} Há uma correlação negativa. Quanto mais armazenamento o comprador pedia, menor era o seu "ganho" percebido.
\end{enumerate}
O ponto crucial é que \textbf{todos os valores \texttt{value\_fn} são altamente positivos}, muito longe do limiar de "perda".

\subsection{Gráfico 3: A Causa Raiz - Distribuição de Parâmetros}

A causa para este mercado "perfeito" está na inicialização dos parâmetros via \texttt{rng.gen\_range}, que cria uma \textbf{distribuição uniforme}. No mundo real, os parâmetros não são uniformes.

\begin{figure}[H]
    \centering
    \includegraphics[width=\textwidth]{Teoria_deModulacao/Teoria_comportamental/parameter_distribuition.png}
    \caption{Análise da Distribuição dos Parâmetros (Todos Uniformes).}
    \label{fig:distributions}
\end{figure}

\textbf{Conclusão do Gráfico 3:} Como os gráficos mostram, todas as suas variáveis de entrada são \textbf{distribuídas uniformemente}. Não há "agrupamentos". Os preços estão espalhados, as reputações estão espalhadas, e os orçamentos estão espalhados. Este tipo de distribuição torna altamente provável a ocorrência de outliers "perfeitos" (como o Provedor 1) e não simula um mercado realista.

\section{Conclusão Definitiva e Próximos Passos}

\textbf{Conclusão:} A implementação da fórmula da Teoria da Perspectiva está correta. No entanto, os \textbf{parâmetros de entrada uniformes} e a \textbf{ausência de restrições de mercado} (como capacidade limitada ou orçamentos mais apertados) criaram um cenário "racional" de "vencedor-leva-tudo", que impediu o surgimento de comportamentos de aversão à perda.

\textbf{Ações Recomendadas (para fazer a PT emergir):}
\begin{enumerate}
    \item \textbf{Implementar Distribuições Estocásticas:}
    \begin{itemize}
        \item \textbf{Orçamentos (\texttt{ref\_point}) e Capacidades (\texttt{capacity}):} Use a \textbf{Distribuição Log-Normal} (ex: \texttt{numpy.random.lognormal}). Isso criará muitos compradores/provedores pequenos e alguns grandes (cauda longa).
        \item \textbf{Preços (\texttt{price}) e Reputação (\texttt{reputation}):} Use a \textbf{Distribuição Normal Truncada} ou \textbf{Distribuição Beta}. Isso agrupará a maioria dos provedores em torno de uma média, tornando a competição mais acirrada.
    \end{itemize}
    \item \textbf{Apertar o Ponto de Referência (Orçamento):} O maior problema é que os orçamentos são muito altos. Um comprador não sentirá "perda" se o preço for trivial. Tente vincular o orçamento ao armazenamento:
    \begin{itemize}
        \item $\text{preço\_esperado\_mercado} = 0.15$
        \item $\text{ref\_point\_base} = \text{storage} \times \text{preço\_esperado\_mercado}$
        \item $\text{ref\_point\_final} = \text{ref\_point\_base} \times (1.0 + \text{variação\_aleatória\_normal})$
    \end{itemize}
    Dessa forma, o \texttt{ref\_point} de um comprador estará "próximo" do preço de mercado, forçando a Teoria da Perspectiva a avaliar "ganhos" e "perdas" pequenos.
\end{enumerate}

    \section{Iteração 2: Ajustes para a Teoria Comportamental Avançada}

    Após a análise inicial (Seção 1), identificamos que a geração de parâmetros uniformes e a ausência de restrições relevantes induziam um mercado "racional" com um provedor dominante. Implementamos as seguintes mudanças no gerador de dados (código Rust em \texttt{src/bin/comportamental\_simulacao.rs}) para induzir comportamentos consistentes com a Teoria da Perspectiva.

    \subsection{Mudanças Principais no Modelo}
    \begin{itemize}
        \item \textbf{Orçamento ancorado ao preço de mercado:} Para cada comprador, o ponto de referência passou a ser calculado como
        \[ \text{ref\_point} = \text{storage} \times p_{\text{mercado}} \times (1 + \epsilon), \quad \epsilon \sim \mathcal{N}(0, 0{,}15), \]
        com $p_{\text{mercado}} = 0{,}15$ R$/GB$ e \(\epsilon\) truncado em $\pm 40\%$. Isso aproxima orçamentos do nível de preços praticado, permitindo ganhos e perdas marginalmente percebidos.

        \item \textbf{Distribuições estocásticas realistas:}
        \begin{itemize}
            \item \textbf{Storage (GB):} Log-normal (mediana \~50 GB), limitando em $[10, 400]$ GB.
            \item \textbf{Capacity (GB):} Log-normal (mediana \~1200 GB), limitando em $[400, 8000]$ GB.
            \item \textbf{Price (R$/GB$):} Normal com média $0{,}15$ e $\sigma=0{,}03$, truncado em $[0{,}10, 0{,}22]$.
            \item \textbf{Reputation:} Beta$(5, 2)$ mapeada para o intervalo $[0{,}4, 0{,}9]$.
            \item \textbf{Fairness Threshold:} Normal ao redor de $0{,}15$ (truncado em $[0{,}10, 0{,}20]$).
        \end{itemize}

        \item \textbf{Restrição de capacidade e alocação:} A capacidade do provedor passou a ser consumida a cada match aceita (\textit{capacity remaining}). Compradores que excedam a capacidade remanescente de um provedor não podem ser atendidos por ele.

        \item \textbf{Critério de aceitação mais restritivo:} A decisão de aceitar requer agora \(v(x) > 0\) (ganho percebido) e pontuação de \textit{fairness} $> 0{,}25$, o que reduz aceitação universal e aumenta a sensibilidade a preços e reputação.
    \end{itemize}

    \subsection{Impacto Esperado}
    As mudanças reduzem a probabilidade de um provedor único dominar consistentemente, elevam a competição e criam casos de perdas percebidas (\(v(x) < 0\)) quando preços superam o ponto de referência. Espera-se observar:
    \begin{itemize}
        \item Taxas de aceitação $< 100\%$;
        \item Distribuição de escolhas entre múltiplos provedores;
        \item Curvas de valor \(\text{value\_fn}\) menos saturadas no regime positivo.
    \end{itemize}

    \subsection{Novos Gráficos (v2)}
    Foram adicionados gráficos com o mesmo visual do arquivo original (\texttt{Graficos.py}) e salvos em \textbf{\texttt{graficos\_analise\_comportamental\_v2}}:
    \begin{itemize}
        \item \textbf{provider\_landscape\_v2.png}: Preço vs. Reputação dos provedores (cenário não-uniforme).
        \item \textbf{buyer\_perception\_v2.png}: Regressões \(\text{value\_fn}\) vs. \texttt{ref\_point} e vs. \texttt{storage}.
        \item \textbf{parameter\_distributions\_v2.png}: Histograma das novas distribuições (\texttt{ref\_point}, \texttt{storage}, \texttt{price}, \texttt{reputation}).
        \item \textbf{accepted\_per\_provider\_v2.png}: \# de matches aceitas por provedor; arquivo \texttt{summary.txt} contém a taxa de aceitação agregada.
    \end{itemize}

    \noindent Os scripts Python correspondentes foram mantidos com o estilo do Seaborn e exportam as figuras automaticamente a partir de \texttt{comportamental\_resultados.json}.

    \subsection{Panorama dos efeitos observados na Iteração 2}
    Após aplicar as alterações da Iteração 2, o comportamento agregado passou a refletir um mercado menos trivial. Em execução representativa, a taxa de aceitação agregada caiu para aproximadamente \textbf{36\%}, com rejeições mais frequentes especialmente entre compradores que demandam grandes volumes de armazenamento. Abaixo sintetizamos os mecanismos que explicam os padrões vistos nos dados e nos gráficos v2:

    \paragraph{Tendência negativa (quase linear) do valor percebido com o aumento do armazenamento.}
    Ao ancorar o orçamento do comprador no preço de mercado, temos \(\text{ref\_point} \approx \text{storage} \times p_{\text{mercado}} \times (1+\epsilon)\), enquanto o custo total é \(\text{storage} \times \text{price}\). Quando o piso de preço é relativamente alto (p.ex., \(\text{price} \geq 0{,}16\) R\$/GB) e o orçamento tem leve viés negativo (\(\epsilon<0\) em média), a diferença por GB tende a ficar negativa. Como essa diferença é multiplicada por \(\text{storage}\), o \textit{resultado relativo} decresce aproximadamente de forma linear com o aumento de armazenamento. Pela Teoria da Perspectiva, para perdas (\(x<0\)) a função de valor é \(v(x)=-\lambda(-x)^{\beta}\) (com \(\beta=0{,}88\)), o que preserva a tendência monotônica: quanto maior o gap negativo total, mais negativo o \(v(x)\). Assim, compradores que solicitam muito armazenamento tendem a perceber perdas e rejeitar mais.

    \paragraph{Parâmetros concentrados nas faixas baixas: por que ocorrem "acúmulos no início".}
    A adoção de \textbf{log-normal} para \textit{storage} e \textit{capacity} cria assimetria com cauda longa: muitos agentes pequenos e poucos grandes. Isso se manifesta como concentração de observações nas faixas iniciais (próximas do mínimo), exatamente o comportamento esperado em mercados reais com alta heterogeneidade e poucos outliers. Esse desenho reduz a probabilidade de todos os compradores terem orçamentos muito folgados simultaneamente.

    \paragraph{Reputação mais alta e menos distribuída.}
    O uso de \(\text{Beta}(5,2)\) mapeada para \([0{,}4,0{,}9]\) é propositalmente \textit{otimista}: a massa de probabilidade se desloca para valores altos, concentrando reputações na ponta superior. Isso reflete um cenário em que a maioria dos provedores tem desempenho adequado (não há grande dispersão de qualidade), restando ao preço e à \textit{fairness} diferenciarem ofertas. Embora tal concentração reduza a heterogeneidade de reputação, ela também amplifica o papel do preço relativo ao limiar de fairness na decisão final.

    \paragraph{Preço por GB e sua interação com reputação e fairness.}
    A distribuição de preço por GB passou a ser normal truncada com piso (e média) acima do preço de mercado, gerando muitos valores próximos ao limite inferior (efeito de truncamento) e alguns mais caros. Como a reputação é elevada, o fator decisivo migra para a \textit{fairness}: a penalidade cresce quando \(\text{price} > \text{fairness\_threshold}\), reduzindo a pontuação \(\text{fairness\_score}=(1-\text{penalty})\times \text{reputation}\). Com piso de preço acima ou próximo do limiar, mesmo provedores bem avaliados podem falhar no corte (p.ex., \(\text{fairness\_score} > 0{,}40\)). Em contrapartida, provedores com preço perto do piso tendem a concentrar as aceitações. Essa dinâmica explica por que, apesar de reputações altas e relativamente próximas, observamos rejeições substanciais e seleção em torno de ofertas mais \textit{baratas e justas}.

    \paragraph{Síntese.}
    A combinação de (i) orçamentos ancorados ao mercado com leve viés negativo, (ii) preços com piso acima do mercado, (iii) reputações altas porém pouco discriminantes e (iv) um critério de fairness mais exigente, produz exatamente o padrão observado: quedas de \(v(x)\) com o aumento do armazenamento, rejeições frequentes quando o gap por GB é desfavorável, e aceitações concentradas nas ofertas com melhor razão preço/limiar de fairness. O resultado final é uma taxa de aceitação significativamente menor do que 100\%, com decisões pautadas por pequenas variações ao redor de um patamar de mercado, onde a Teoria da Perspectiva efetivamente se manifesta.

% --- FIM DO DOCUMENTO ---
\end{document}